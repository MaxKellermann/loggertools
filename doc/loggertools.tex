\documentclass{article}

\usepackage{hyperref}

\title{loggertools}
\author{Max Kellermann}

\begin{document}
\maketitle
\tableofcontents


\section{Abstract}

{\em loggertools} is a collection of free tools and libraries which
talk to GPS flight loggers.

The project is aimed at users of alternative operating systems like
Linux and Mac OS X, who usually receive no support from the hardware
vendors.


\section{Installing {\em loggertools}}

To install {\em loggertools}, you need GNU make and gcc, plus the
\href{http://www.boost.org/}{Boost libraries}.  Download the {\em
loggertools} source tarball:

\begin{verbatim}
wget http://max.kellermann.name/download/loggertools/0.x/loggertools-0.1.0.tar.gz
tar xzf loggertools-0.1.0.tar.gz
cd loggertools-0.1.0
make
make install
\end{verbatim}

The last step must be done as {\em root}.


\section{Tools for Filser / LX Navigation devices}

\subsection{{\em lxn-logger}: Downloading flights from IGC loggers}

The program {\em lxn-logger} can download flights from IGC loggers
like the Colibri.  When you start the program, it contacts the device
({\em /dev/ttyS0} by default) and presents you a list of all flights.
You will be prompted to enter the number of the flights which should
be downloaded.

After a successful download, the current directory will contain the
flight file in {\em LXN} and {\em IGC} format.  The latter is the
interesting one, which is required by competitions like the Online
Contest (OLC).

\subsection{{\em lxn2igc}: file converter}

This is a small utility which converts {\em LXN} files to the {\em
IGC} format.  Its code is contained in {\em lxn-logger}, so normally
you will not need it.


\section{Tools for Holltronic (Cenfis) devices}

\subsection{{\em cenfistool}}

{\em cenfistool} can upload files in the Hexfile format to the Cenfis
device.  The hexfiles may contain firmware upgrades, turnpoints or
airspace definitions.

To perform a firmware upgrade, you have to press the dark green and
the red button while switching the Cenfis on.  Read the instructions
on the screen carefully and start {\em cenfistool} with the ``upload''
command.

Turnpoint and airspace uploads are launched in the ``red button
menu''.

\subsection{{\em hexfile}}

This program can encode files into the Hexfile format, more precisely
the Hexfile dialect suitable for the Cenfis, which requires bank
switching for 32 kB banks.

In the following sample, {\em hexfile} decodes the firmware image into
binary:

\begin{verbatim}
hexfile -d -o cfp_v37c.bin cfp_v37c.bhf
\end{verbatim}


\section{Database converters}

There are several free-of-charge databases on the net, like
\href{http://www.segelflug.de/segelflieger/michael.meier/}{Open-Air},
and {\em loggertools} provides the tools to work with these databases.

Most formats are not documented and had to be reverse engineered.


\subsection{{\em tpconv}: Turn point converter}

{\em tpconv} can convert turn point databases between several formats
and supports filtering.

Some formats are richer than others.  Keep in mind that information
can be lost during conversion.

The following formats are supported:

\begin{tabular}{|l|l|}
\hline
SeeYou & *.cup \\
\hline
Cenfis & *.cdb, *.idb, *.dab, *.bhf \\
\hline
Filser / LXN & *.da4 \\
\hline
Zander & *.wz \\
\hline
\end{tabular}

To convert the SeeYou file {\em TurnPoints.cup} to a Cenfis Hexfile,
enter:

\begin{verbatim}
tpconv TurnPoints.cup -o TurnPoints.bhf 
\end{verbatim}


\subsection{{\em asconv}: Airspace converter}

This tool is under development and not yet usable.


\end{document}
